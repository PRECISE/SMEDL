\documentclass[paper=letter,titlepage,captions=tableheading]{scrartcl}
\usepackage{lmodern}
\usepackage[T1]{fontenc}
\usepackage{textcomp}
\usepackage{ctable}
\usepackage{listings}
\usepackage{xcolor}
\usepackage{hyperref}
\usepackage{scrhack}
%TODO check the options for hyperref and add useful ones

% Allow bold teletype text
\DeclareFontFamily{T1}{lmtt}{}
\DeclareFontShape{T1}{lmtt}{m}{n}{<-> ec-lmtl10}{}
\DeclareFontShape{T1}{lmtt}{m}{\itdefault}{<-> ec-lmtlo10}{}
\DeclareFontShape{T1}{lmtt}{\bfdefault}{n}{<-> ec-lmtk10}{}
\DeclareFontShape{T1}{lmtt}{\bfdefault}{\itdefault}{<-> ec-lmtko10}{}

% No dot after table and figure names in captions
% e.g. "Table 1.1:", not "Table 1.1.:"
\renewcommand*{\figureformat}{\figurename~\thefigure}
\renewcommand*{\tableformat}{\tablename~\thetable}

\definecolor{ltGray}{HTML}{F5F5F5}
\definecolor{dkRed}{HTML}{A52A2A}
\definecolor{dkBlue}{HTML}{483D8B}
\definecolor{dkGreen}{HTML}{008070}
\definecolor{dkYellow}{HTML}{B8860B}
\definecolor{slate}{HTML}{778899}

%\newcommand{\textcolordummy}[2]{#2}

\lstset{
  numbers=left,
  xleftmargin=1em,
  frame=single,
  frameround=tttt,
  belowcaptionskip=3ex,
  backgroundcolor=\color{ltGray},
  basicstyle=\ttfamily,
  basewidth={0.5em,0.5em},
  numberstyle=\color{slate}\tiny %line numbers
}

\lstdefinestyle{fixed}{
  aboveskip=1em,
  belowskip=1em
}

\lstdefinelanguage{smedl}{
  morekeywords={object,state:,events:,scenarios:,imported,internal,exported,when,else,raise,true,false,null,NULL},
  morekeywords=[2]{int,float,double,char,string,pointer,thread,opaque},
  morekeywords=[3]{\#include},
  alsoletter={\#},
  alsodigit={:},
  sensitive=true,
  morecomment=[l]{//},
  morecomment=[s]{/*}{*/},
  morestring=[b]"
}

\lstdefinestyle{smedlbase}{
  language=smedl,
  %commentstyle={\color{slate}\let\textcolor\textcolordummy},
  commentstyle=\color{slate}\itshape,
  keywordstyle=\color{dkBlue}\bfseries,
  keywordstyle=[2]\color{dkGreen}\bfseries,
  keywordstyle=[3]\color{dkYellow}\bfseries,
  %directivestyle=\color{dkYellow}\bfseries,
  stringstyle=\color{dkRed},
  literate=*{0}{{\textcolor{dkRed}{0}}}{1}%
            {1}{{\textcolor{dkRed}{1}}}{1}%
            {2}{{\textcolor{dkRed}{2}}}{1}%
            {3}{{\textcolor{dkRed}{3}}}{1}%
            {4}{{\textcolor{dkRed}{4}}}{1}%
            {5}{{\textcolor{dkRed}{5}}}{1}%
            {6}{{\textcolor{dkRed}{6}}}{1}%
            {7}{{\textcolor{dkRed}{7}}}{1}%
            {8}{{\textcolor{dkRed}{8}}}{1}%
            {9}{{\textcolor{dkRed}{9}}}{1}%
            {.0}{{\textcolor{dkRed}{.0}}}{2}%
            {.1}{{\textcolor{dkRed}{.1}}}{2}%
            {.2}{{\textcolor{dkRed}{.2}}}{2}%
            {.3}{{\textcolor{dkRed}{.3}}}{2}%
            {.4}{{\textcolor{dkRed}{.4}}}{2}%
            {.5}{{\textcolor{dkRed}{.5}}}{2}%
            {.6}{{\textcolor{dkRed}{.6}}}{2}%
            {.7}{{\textcolor{dkRed}{.7}}}{2}%
            {.8}{{\textcolor{dkRed}{.8}}}{2}%
            {.9}{{\textcolor{dkRed}{.9}}}{2}%
            {+0}{{\textcolor{dkRed}{+0}}}{2}%
            {+1}{{\textcolor{dkRed}{+1}}}{2}%
            {+2}{{\textcolor{dkRed}{+2}}}{2}%
            {+3}{{\textcolor{dkRed}{+3}}}{2}%
            {+4}{{\textcolor{dkRed}{+4}}}{2}%
            {+5}{{\textcolor{dkRed}{+5}}}{2}%
            {+6}{{\textcolor{dkRed}{+6}}}{2}%
            {+7}{{\textcolor{dkRed}{+7}}}{2}%
            {+8}{{\textcolor{dkRed}{+8}}}{2}%
            {+9}{{\textcolor{dkRed}{+9}}}{2}%
            {-0}{{\textcolor{dkRed}{-0}}}{2}%
            {-1}{{\textcolor{dkRed}{-1}}}{2}%
            {-2}{{\textcolor{dkRed}{-2}}}{2}%
            {-3}{{\textcolor{dkRed}{-3}}}{2}%
            {-4}{{\textcolor{dkRed}{-4}}}{2}%
            {-5}{{\textcolor{dkRed}{-5}}}{2}%
            {-6}{{\textcolor{dkRed}{-6}}}{2}%
            {-7}{{\textcolor{dkRed}{-7}}}{2}%
            {-8}{{\textcolor{dkRed}{-8}}}{2}%
            {-9}{{\textcolor{dkRed}{-9}}}{2}%
            {e0}{{\textcolor{dkRed}{e0}}}{2}%
            {e1}{{\textcolor{dkRed}{e1}}}{2}%
            {e2}{{\textcolor{dkRed}{e2}}}{2}%
            {e3}{{\textcolor{dkRed}{e3}}}{2}%
            {e4}{{\textcolor{dkRed}{e4}}}{2}%
            {e5}{{\textcolor{dkRed}{e5}}}{2}%
            {e6}{{\textcolor{dkRed}{e6}}}{2}%
            {e7}{{\textcolor{dkRed}{e7}}}{2}%
            {e8}{{\textcolor{dkRed}{e8}}}{2}%
            {e9}{{\textcolor{dkRed}{e9}}}{2}%
            {e+0}{{\textcolor{dkRed}{e-0}}}{3}%
            {e+1}{{\textcolor{dkRed}{e-1}}}{3}%
            {e+2}{{\textcolor{dkRed}{e-2}}}{3}%
            {e+3}{{\textcolor{dkRed}{e-3}}}{3}%
            {e+4}{{\textcolor{dkRed}{e-4}}}{3}%
            {e+5}{{\textcolor{dkRed}{e-5}}}{3}%
            {e+6}{{\textcolor{dkRed}{e-6}}}{3}%
            {e+7}{{\textcolor{dkRed}{e-7}}}{3}%
            {e+8}{{\textcolor{dkRed}{e-8}}}{3}%
            {e+9}{{\textcolor{dkRed}{e-9}}}{3}%
            {e-0}{{\textcolor{dkRed}{e-0}}}{3}%
            {e-1}{{\textcolor{dkRed}{e-1}}}{3}%
            {e-2}{{\textcolor{dkRed}{e-2}}}{3}%
            {e-3}{{\textcolor{dkRed}{e-3}}}{3}%
            {e-4}{{\textcolor{dkRed}{e-4}}}{3}%
            {e-5}{{\textcolor{dkRed}{e-5}}}{3}%
            {e-6}{{\textcolor{dkRed}{e-6}}}{3}%
            {e-7}{{\textcolor{dkRed}{e-7}}}{3}%
            {e-8}{{\textcolor{dkRed}{e-8}}}{3}%
            {e-9}{{\textcolor{dkRed}{e-9}}}{3}%
            {->}{{\textcolor{dkYellow}{->}}}{2}
}

\lstdefinestyle{smedl}{
  style=smedlbase,
  float
}

\lstdefinestyle{smedlfixed}{
  style=smedlbase,
  style=fixed
}

\lstdefinelanguage{a4smedl}{
  morekeywords={system,import,monitor,as,event,syncset},
  morekeywords=[2]{int,float,double,char,string,pointer,thread,opaque},
  morekeywords=[3]{Id.,Param.},
  alsodigit={.},
  sensitive=true,
  morecomment=[l]{//},
  morecomment=[s]{/*}{*/},
  morestring=[b]"
}

\lstdefinestyle{a4smedlbase}{
  language=a4smedl,
  commentstyle=\color{slate}\itshape,
  keywordstyle=\color{dkBlue}\bfseries,
  keywordstyle=[2]\color{dkGreen}\bfseries,
  keywordstyle=[3]\color{dkRed},
  stringstyle=\color{dkRed},
  literate=*{0}{{\textcolor{dkRed}{0}}}{1}%
            {1}{{\textcolor{dkRed}{1}}}{1}%
            {2}{{\textcolor{dkRed}{2}}}{1}%
            {3}{{\textcolor{dkRed}{3}}}{1}%
            {4}{{\textcolor{dkRed}{4}}}{1}%
            {5}{{\textcolor{dkRed}{5}}}{1}%
            {6}{{\textcolor{dkRed}{6}}}{1}%
            {7}{{\textcolor{dkRed}{7}}}{1}%
            {8}{{\textcolor{dkRed}{8}}}{1}%
            {9}{{\textcolor{dkRed}{9}}}{1}%
            {$0}{{\textcolor{dkRed}{\$0}}}{2}%
            {$1}{{\textcolor{dkRed}{\$1}}}{2}%
            {$2}{{\textcolor{dkRed}{\$2}}}{2}%
            {$3}{{\textcolor{dkRed}{\$3}}}{2}%
            {$4}{{\textcolor{dkRed}{\$4}}}{2}%
            {$5}{{\textcolor{dkRed}{\$5}}}{2}%
            {$6}{{\textcolor{dkRed}{\$6}}}{2}%
            {$7}{{\textcolor{dkRed}{\$7}}}{2}%
            {$8}{{\textcolor{dkRed}{\$8}}}{2}%
            {$9}{{\textcolor{dkRed}{\$9}}}{2}%
            {\#0}{{\textcolor{dkRed}{\#0}}}{2}%
            {\#1}{{\textcolor{dkRed}{\#1}}}{2}%
            {\#2}{{\textcolor{dkRed}{\#2}}}{2}%
            {\#3}{{\textcolor{dkRed}{\#3}}}{2}%
            {\#4}{{\textcolor{dkRed}{\#4}}}{2}%
            {\#5}{{\textcolor{dkRed}{\#5}}}{2}%
            {\#6}{{\textcolor{dkRed}{\#6}}}{2}%
            {\#7}{{\textcolor{dkRed}{\#7}}}{2}%
            {\#8}{{\textcolor{dkRed}{\#8}}}{2}%
            {\#9}{{\textcolor{dkRed}{\#9}}}{2}%
            {=>}{{\textcolor{dkYellow}{=>}}}{2}
}

\lstdefinestyle{a4smedl}{
  style=a4smedlbase,
  float
}

\lstdefinestyle{a4smedlfixed}{
  style=a4smedlbase,
  style=fixed
}

\lstdefinestyle{Cbase}{
  language=C,
  commentstyle=\color{slate}\itshape,
  keywordstyle=\color{dkBlue}\bfseries,
  keywordstyle=[2]\color{dkGreen}\bfseries,
  keywordstyle=[3]\color{dkRed},
  stringstyle=\color{dkRed},
  literate=*{0}{{\textcolor{dkRed}{0}}}{1}%
            {1}{{\textcolor{dkRed}{1}}}{1}%
            {2}{{\textcolor{dkRed}{2}}}{1}%
            {3}{{\textcolor{dkRed}{3}}}{1}%
            {4}{{\textcolor{dkRed}{4}}}{1}%
            {5}{{\textcolor{dkRed}{5}}}{1}%
            {6}{{\textcolor{dkRed}{6}}}{1}%
            {7}{{\textcolor{dkRed}{7}}}{1}%
            {8}{{\textcolor{dkRed}{8}}}{1}%
            {9}{{\textcolor{dkRed}{9}}}{1}%
            {.0}{{\textcolor{dkRed}{.0}}}{2}%
            {.1}{{\textcolor{dkRed}{.1}}}{2}%
            {.2}{{\textcolor{dkRed}{.2}}}{2}%
            {.3}{{\textcolor{dkRed}{.3}}}{2}%
            {.4}{{\textcolor{dkRed}{.4}}}{2}%
            {.5}{{\textcolor{dkRed}{.5}}}{2}%
            {.6}{{\textcolor{dkRed}{.6}}}{2}%
            {.7}{{\textcolor{dkRed}{.7}}}{2}%
            {.8}{{\textcolor{dkRed}{.8}}}{2}%
            {.9}{{\textcolor{dkRed}{.9}}}{2}%
            {+0}{{\textcolor{dkRed}{+0}}}{2}%
            {+1}{{\textcolor{dkRed}{+1}}}{2}%
            {+2}{{\textcolor{dkRed}{+2}}}{2}%
            {+3}{{\textcolor{dkRed}{+3}}}{2}%
            {+4}{{\textcolor{dkRed}{+4}}}{2}%
            {+5}{{\textcolor{dkRed}{+5}}}{2}%
            {+6}{{\textcolor{dkRed}{+6}}}{2}%
            {+7}{{\textcolor{dkRed}{+7}}}{2}%
            {+8}{{\textcolor{dkRed}{+8}}}{2}%
            {+9}{{\textcolor{dkRed}{+9}}}{2}%
            {-0}{{\textcolor{dkRed}{-0}}}{2}%
            {-1}{{\textcolor{dkRed}{-1}}}{2}%
            {-2}{{\textcolor{dkRed}{-2}}}{2}%
            {-3}{{\textcolor{dkRed}{-3}}}{2}%
            {-4}{{\textcolor{dkRed}{-4}}}{2}%
            {-5}{{\textcolor{dkRed}{-5}}}{2}%
            {-6}{{\textcolor{dkRed}{-6}}}{2}%
            {-7}{{\textcolor{dkRed}{-7}}}{2}%
            {-8}{{\textcolor{dkRed}{-8}}}{2}%
            {-9}{{\textcolor{dkRed}{-9}}}{2}%
            {e0}{{\textcolor{dkRed}{e0}}}{2}%
            {e1}{{\textcolor{dkRed}{e1}}}{2}%
            {e2}{{\textcolor{dkRed}{e2}}}{2}%
            {e3}{{\textcolor{dkRed}{e3}}}{2}%
            {e4}{{\textcolor{dkRed}{e4}}}{2}%
            {e5}{{\textcolor{dkRed}{e5}}}{2}%
            {e6}{{\textcolor{dkRed}{e6}}}{2}%
            {e7}{{\textcolor{dkRed}{e7}}}{2}%
            {e8}{{\textcolor{dkRed}{e8}}}{2}%
            {e9}{{\textcolor{dkRed}{e9}}}{2}%
            {e+0}{{\textcolor{dkRed}{e-0}}}{3}%
            {e+1}{{\textcolor{dkRed}{e-1}}}{3}%
            {e+2}{{\textcolor{dkRed}{e-2}}}{3}%
            {e+3}{{\textcolor{dkRed}{e-3}}}{3}%
            {e+4}{{\textcolor{dkRed}{e-4}}}{3}%
            {e+5}{{\textcolor{dkRed}{e-5}}}{3}%
            {e+6}{{\textcolor{dkRed}{e-6}}}{3}%
            {e+7}{{\textcolor{dkRed}{e-7}}}{3}%
            {e+8}{{\textcolor{dkRed}{e-8}}}{3}%
            {e+9}{{\textcolor{dkRed}{e-9}}}{3}%
            {e-0}{{\textcolor{dkRed}{e-0}}}{3}%
            {e-1}{{\textcolor{dkRed}{e-1}}}{3}%
            {e-2}{{\textcolor{dkRed}{e-2}}}{3}%
            {e-3}{{\textcolor{dkRed}{e-3}}}{3}%
            {e-4}{{\textcolor{dkRed}{e-4}}}{3}%
            {e-5}{{\textcolor{dkRed}{e-5}}}{3}%
            {e-6}{{\textcolor{dkRed}{e-6}}}{3}%
            {e-7}{{\textcolor{dkRed}{e-7}}}{3}%
            {e-8}{{\textcolor{dkRed}{e-8}}}{3}%
            {e-9}{{\textcolor{dkRed}{e-9}}}{3}
}

\lstdefinestyle{C}{
  style=Cbase,
  float
}

\lstdefinestyle{Cfixed}{
  style=Cbase,
  style=fixed
}



\title{SMEDL Manual}
\author{University of Pennsylvania}

\begin{document}

\maketitle

%TODO Basic introduction to SMEDL

\autoref{part:lang} of this manual describes how to write a monitor. It discusses the features and syntax of the SMEDL specification language. \autoref{part:api} is about integrating monitors into your existing code. It describes the API for the generated code and the various transport mechanisms available for asynchronous connections.

\clearpage
\part{The SMEDL and A4SMEDL Monitoring Language}
\label{part:lang}

A SMEDL monitoring system is designed at two levels: the monitor level and
the architecture level. Monitor specifications describe a single monitor, that
is, a list of state variables, a list of events that the monitor consumes and
emits (known as ``imported'' and ``exported'' events), and a set of states,
transitions, and actions describing the state machine. Architecture
specifications describe how multiple monitors come together to form a
monitoring system, including which monitor specifications are involved, how
instances of the monitors are parameterized, and how imported and exported
events are directed.

Monitor specifications are written in \verb|.smedl| files, and there may be
multiple in a monitoring system. Architecture specifications are written in
\verb|.a4smedl| files and there is exactly one per monitoring system.

\section{SMEDL Specifications}
\label{sec:smedl}

A SMEDL specification describes a single monitor. This is where the actual
monitoring logic is defined. At its core, a SMEDL monitor is a set of state
machines. Transitions are triggered by events, with optional conditions and
associated actions. Monitors may also contain state variables, which the
conditions and actions can make use of.

\subsection{A Basic Monitor}
\label{subsec:smedl-basic}

\begin{lstlisting}[
  caption={A monitor that keeps a running total},
  label=lst:basic-mon,
  style=smedl
]
object Adder;

state:
  float accumulator = 0;

events:
  imported measurement(float);
  exported sum(float);

scenarios:
  main:
    idle
      -> measurement(val) {
        accumulator = accumulator + val;
        raise sum(accumulator);
      }
      -> idle;
\end{lstlisting}

\autoref{lst:basic-mon} gives an example of a simple monitor. This guide
will break it down piece-by-piece to familiarize you with the basic syntax of
SMEDL.

\begin{lstlisting}[firstnumber=1,style=smedlfixed]
object Adder;
\end{lstlisting}

This line declares the monitor specification name. It must be present at the
top of every monitor specification. The name can be any valid SMEDL identifier.
To be specific, an upper or lowercase letter, followed by zero or more
additional letters, numbers, or underscores.

This is the only line in the file that stands alone. The rest of the specification is split into three main sections, the start of each marked by a heading
keyword.

\begin{lstlisting}[firstnumber=3,style=smedlfixed]
state:
  float accumulator = 0;
\end{lstlisting}

This is the state variables section. State variables can be thought of as
global variables for the monitor. They will be useful for conditions and
actions (each of which are discussed later). There are seven types (and one
alias) available, listed in \autoref{tab:types1} along with the most similar C
type for each. This is the only optional section: If there is no need for state
variables, it may be omitted entirely.

\ctable[
  caption={The types available in SMEDL},
  label=tab:types1,
  mincapwidth=3in,
]{ll}{
  \tnote[a]{\texttt{double} is an alias of \texttt{float} for convenience, not a distinct type.}
  \tnote[b]{The \texttt{SMEDLOpaque} type is dicussed in %(TODO section from language reference).
  }
}{
  \toprule
  SMEDL & C \\
  \midrule
  \texttt{int} & \texttt{int} \\
  \texttt{float} & \texttt{double} \\
  \texttt{double}\tmark[a] & \texttt{double} \\
  \texttt{char} & \texttt{char} \\
  \texttt{string} & \texttt{char *} \emph{or} \texttt{char[]} \\
  \texttt{pointer} & \texttt{void *} \\
  \texttt{thread} & \texttt{pthread\_t} \\
  \texttt{opaque} & \texttt{SMEDLOpaque}\tmark[b] \\
  \bottomrule
}

\begin{lstlisting}[firstnumber=6,style=smedlfixed]
events:
  imported measurement(float);
  exported sum(float);
\end{lstlisting}

This is the event declarations section. Events are the input and output of the
monitor. Imported events come from outside the monitor and may trigger
transitions and actions. Exported events are generated within the monitor and
sent out. (There are also internal events, introduced in \autoref{subsec:smedl-advanced}.)

Events can carry parameters with them. TODO

\begin{lstlisting}[firstnumber=10,style=smedlfixed]
scenarios:
  main:
    idle
      -> measurement(val) {
        accumulator = accumulator + val;
        raise sum(accumulator);
      }
      -> idle;
\end{lstlisting}

\subsection{A More Advanced Monitor}
\label{subsec:smedl-advanced}

%TODO Use the frontend monitor from multi_moving_average, or maybe a monitor
% for LTL. Actually, the latter is a good idea because it still makes sense
% as standalone. Go over more advanced features, like else, multiple scenarios,
% and internal events. Mention details like scenarios can only execute once per
% macro step and exported events can also act as internal events.

\subsection{SMEDL Language Reference}
\label{subsec:smedl-ref}

%TODO Start by describing the identifier format, then go into types, then
% go through sections one by one


%NOTE In an expression, == on strings normally compares the contents of the
% strings, but due to limitations in type verification, this does not happen
% when both sides of the == or != are helper functions that return strings.
% Similar for opaques.

%NOTE Opaque type does not allow any operations except == and !=. These
% operations do a byte-by-byte comparison with memcmp. If that is not
% acceptable, an alternative is to use a helper function that accepts two
% SMEDLOpaque and returns zero or nonzero.

\section{A4SMEDL Specifications}
\label{sec:a4smedl}

\subsection{A Basic Architecture}
\label{subsec:a4smedl-basic}

\subsection{A More Advanced Architecture}
\label{subsec:a4smedl-advanced}

\subsection{A4SMEDL Language Reference}
\label{subsec:a4smedl-ref}

%TODO Show a basic example going over the most common features. Then have sections on other features (like dynamic instantiation). Finish with a full reference (maybe in a separate chapter). Actually probably do this in sections.

%NOTE Using floats as identity parameters is experimental at best and should
% very likely be avoided. Hashing may not satisfy the invariant that if a == b,
% hash(a) == hash(b). And floats must compare exactly equal.
%NOTE Actually, I think the hashing is no longer a problem

%NOTE Using threads as identity parameters may result in undefined behavior.
% There is no portable way to compare `pthread_t`.

%NOTE See note in prev section on opaques. If they cannot be tested for equality
% with memcmp, then it is not safe to use them as monitor identities.

\clearpage
\part{The SMEDL Programming Interface}
\label{part:api}

The generated monitoring code is separated into multiple layers, ordered here
from lowest to highest:
\begin{description}
  \item[Monitor] contains the actual monitor state machines.
  \item[Local wrapper] manages instances of a single monitor specification and
    dynamic instantiation.
  \item[Global wrapper] manages all local wrappers in a single synchronous set
    and routes events between them.
  \item[Transport] handles asynchronous communication between global wrappers
    and the target system. This layer is only generated when the \texttt{-t}
    option is used with \texttt{mgen}.
\end{description}

The monitor (SMEDL) specification corresponds with the monitor level and the
architecture (A4SMEDL) specification corresponds with the three layers above
it.

\section{Common API Elements}
\label{sec:common}

\textbf{Header: \texttt{smedl\_types.h}}

\subsection{The \texttt{SMEDLValue} Type}
label{subsec:smedlvalue}

The \texttt{SMEDLValue} type is used throughout the API for event parameters,
monitor identities, etc. It is defined in \autoref{lst:smedlvalue}. Each 
\texttt{SMEDLValue} contains a type \texttt{t} and a value \texttt{v}. The type
is a member of the \texttt{SMEDLType} enum and the value is a union. See
\autoref{tab:smedlvalue-union} for the correspondence between the SMEDL types
and union members.

%TODO C listings aren't quite formatted right...at the very least, keywords are
% not bold
\begin{lstlisting}[
  caption={The \texttt{SMEDLValue} and \texttt{SMEDLType} definitons},
  label=lst:smedlvalue,
  style=Cfixed
]
typedef enum {SMEDL_INT, SMEDL_FLOAT, SMEDL_CHAR,
    SMEDL_STRING, SMEDL_POINTER, SMEDL_THREAD,
    SMEDL_OPAQUE, SMEDL_NULL} SMEDLType;

typedef struct {
    SMEDLType t;
    union {
        int i;
        double d;
        char c;
        char *s;
        void *p;
        pthread_t th;
        SMEDLOpaque o;
    } v;
} SMEDLValue;
\end{lstlisting}

\ctable[
  caption={Correspondence between \texttt{SMEDLType} and union in
  \texttt{SMEDLValue}},
  label=tab:smedlvalue-union,
  mincapwidth=3in,
]{ll}{}{
  \toprule
  \texttt{SMEDLType} & Union member \\
  \midrule
  \texttt{SMEDL\_INT} & \texttt{i} \\
  \texttt{SMEDL\_FLOAT} & \texttt{d} \\
  \texttt{SMEDL\_CHAR} & \texttt{c} \\
  \texttt{SMEDL\_STRING} & \texttt{s} \\
  \texttt{SMEDL\_POINTER} & \texttt{p} \\
  \texttt{SMEDL\_THREAD} & \texttt{th} \\
  \texttt{SMEDL\_OPAQUE} & \texttt{o} \\
  \texttt{SMEDL\_NULL} & -- \\
  \bottomrule
}

The \texttt{SMEDL\_NULL} type is special---it does not correspond with an
actual SMEDL type. It is used internally in a couple places, but as far as the
API is concerned, there is only one significant use: wildcard parameters. In a
list of monitor identities, a \texttt{SMEDLValue} with \texttt{t} set to
\texttt{SMEDL\_NULL} is understood to be a wildcard. The \texttt{v} member is
ignored in such cases.

\subsection{The \texttt{SMEDLOpaque} Type}
\label{subsec:smedlopaque}

\texttt{SMEDLOpaque} \texttt{\bfseries SMEDLOpaque}

%TODO SMEDLValue and certain other things

%NOTE Strings and opaques cannot point to NULL. Strings must always be
% null-terminated.

\section{Monitor API}
\label{sec:mon}

%TODO

\section{Local Wrapper API}
\label{sec:local}

%TODO

\section{Global Wrapper API}
\label{sec:global}

%TODO

\section{Transport Adapters}
\label{sec:transport}

%TODO

\subsection{RabbitMQ}
\label{subsec:rabbitmq}

%TODO

%\subsection{ROS}
%\label{subsec:ros}
%TODO

\subsection{File}
\label{subsec:file}

%TODO

%\section{PEDL API}
%\label{sec:pedl}
% At some point we may have an additional interface specifically designed for
% the target system to interface with. It would have functions named after the
% imported events (instead of channel names) and these functions would either
% call the global wrapper import API directly or generate RabbitMQ/ROS messages
% depending on whether synchronous or asynchronous transport is necessary. Such
% an interface would be documented in a section here.
\end{document}
