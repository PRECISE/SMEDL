\documentclass[paper=letter,titlepage,captions=tableheading]{scrartcl}
\usepackage[lighttt]{lmodern}
\usepackage[T1]{fontenc}
%\usepackage{textcomp}
\usepackage{ctable}
\usepackage{listings}
\usepackage{xcolor}
\usepackage{framed}
\usepackage{hyperref}
\usepackage{scrhack}
%TODO check the options for hyperref and add useful ones

% Workaround for bug with light italic tt
% See https://tex.stackexchange.com/a/234020
\ttfamily
\DeclareFontShape{T1}{lmtt}{m}{it}
     {<->sub*lmtt/m/sl}{}

% No dot after table and figure names in captions
% e.g. "Table 1.1:", not "Table 1.1.:"
\renewcommand*{\figureformat}{\figurename~\thefigure}
\renewcommand*{\tableformat}{\tablename~\thetable}

% Color for shaded boxes
\definecolor{shadecolor}{HTML}{F5F5F5}

\definecolor{ltGray}{HTML}{F5F5F5}
\definecolor{dkRed}{HTML}{A52A2A}
\definecolor{dkBlue}{HTML}{483D8B}
\definecolor{dkGreen}{HTML}{008070}
\definecolor{dkYellow}{HTML}{B8860B}
\definecolor{slate}{HTML}{778899}

%\newcommand{\textcolordummy}[2]{#2}

\lstset{
  numbers=left,
  xleftmargin=1em,
  frame=single,
  frameround=tttt,
  belowcaptionskip=3ex,
  backgroundcolor=\color{ltGray},
  basicstyle=\ttfamily,%
  basewidth={0.5em,0.5em},
  numberstyle=\color{slate}\tiny %line numbers
}

\lstdefinestyle{fixed}{
  aboveskip=1em,
  belowskip=1em
}

\lstdefinelanguage{smedl}{
  morekeywords={object,state:,events:,scenarios:,imported,internal,exported,when,else,raise,true,false,null,NULL},
  morekeywords=[2]{int,float,double,char,string,pointer,thread,opaque},
  morekeywords=[3]{\#include},
  alsoletter={\#},
  alsodigit={:},
  sensitive=true,
  morecomment=[l]{//},
  morecomment=[s]{/*}{*/},
  morestring=[b]"
}

\lstdefinestyle{smedlbase}{%
  language=smedl,
  %commentstyle={\color{slate}\let\textcolor\textcolordummy},
  commentstyle=\color{slate}\itshape,%
  keywordstyle=\color{dkBlue}\bfseries,%
  keywordstyle=[2]\color{dkGreen}\bfseries,%
  keywordstyle=[3]\color{dkYellow}\bfseries,%
  %directivestyle=\color{dkYellow}\bfseries,%
  stringstyle=\color{dkRed},%
  literate=*{0}{{\textcolor{dkRed}{0}}}{1}%
            {1}{{\textcolor{dkRed}{1}}}{1}%
            {2}{{\textcolor{dkRed}{2}}}{1}%
            {3}{{\textcolor{dkRed}{3}}}{1}%
            {4}{{\textcolor{dkRed}{4}}}{1}%
            {5}{{\textcolor{dkRed}{5}}}{1}%
            {6}{{\textcolor{dkRed}{6}}}{1}%
            {7}{{\textcolor{dkRed}{7}}}{1}%
            {8}{{\textcolor{dkRed}{8}}}{1}%
            {9}{{\textcolor{dkRed}{9}}}{1}%
            {.0}{{\textcolor{dkRed}{.0}}}{2}%
            {.1}{{\textcolor{dkRed}{.1}}}{2}%
            {.2}{{\textcolor{dkRed}{.2}}}{2}%
            {.3}{{\textcolor{dkRed}{.3}}}{2}%
            {.4}{{\textcolor{dkRed}{.4}}}{2}%
            {.5}{{\textcolor{dkRed}{.5}}}{2}%
            {.6}{{\textcolor{dkRed}{.6}}}{2}%
            {.7}{{\textcolor{dkRed}{.7}}}{2}%
            {.8}{{\textcolor{dkRed}{.8}}}{2}%
            {.9}{{\textcolor{dkRed}{.9}}}{2}%
            {+0}{{\textcolor{dkRed}{+0}}}{2}%
            {+1}{{\textcolor{dkRed}{+1}}}{2}%
            {+2}{{\textcolor{dkRed}{+2}}}{2}%
            {+3}{{\textcolor{dkRed}{+3}}}{2}%
            {+4}{{\textcolor{dkRed}{+4}}}{2}%
            {+5}{{\textcolor{dkRed}{+5}}}{2}%
            {+6}{{\textcolor{dkRed}{+6}}}{2}%
            {+7}{{\textcolor{dkRed}{+7}}}{2}%
            {+8}{{\textcolor{dkRed}{+8}}}{2}%
            {+9}{{\textcolor{dkRed}{+9}}}{2}%
            {-0}{{\textcolor{dkRed}{-0}}}{2}%
            {-1}{{\textcolor{dkRed}{-1}}}{2}%
            {-2}{{\textcolor{dkRed}{-2}}}{2}%
            {-3}{{\textcolor{dkRed}{-3}}}{2}%
            {-4}{{\textcolor{dkRed}{-4}}}{2}%
            {-5}{{\textcolor{dkRed}{-5}}}{2}%
            {-6}{{\textcolor{dkRed}{-6}}}{2}%
            {-7}{{\textcolor{dkRed}{-7}}}{2}%
            {-8}{{\textcolor{dkRed}{-8}}}{2}%
            {-9}{{\textcolor{dkRed}{-9}}}{2}%
            {e0}{{\textcolor{dkRed}{e0}}}{2}%
            {e1}{{\textcolor{dkRed}{e1}}}{2}%
            {e2}{{\textcolor{dkRed}{e2}}}{2}%
            {e3}{{\textcolor{dkRed}{e3}}}{2}%
            {e4}{{\textcolor{dkRed}{e4}}}{2}%
            {e5}{{\textcolor{dkRed}{e5}}}{2}%
            {e6}{{\textcolor{dkRed}{e6}}}{2}%
            {e7}{{\textcolor{dkRed}{e7}}}{2}%
            {e8}{{\textcolor{dkRed}{e8}}}{2}%
            {e9}{{\textcolor{dkRed}{e9}}}{2}%
            {e+0}{{\textcolor{dkRed}{e-0}}}{3}%
            {e+1}{{\textcolor{dkRed}{e-1}}}{3}%
            {e+2}{{\textcolor{dkRed}{e-2}}}{3}%
            {e+3}{{\textcolor{dkRed}{e-3}}}{3}%
            {e+4}{{\textcolor{dkRed}{e-4}}}{3}%
            {e+5}{{\textcolor{dkRed}{e-5}}}{3}%
            {e+6}{{\textcolor{dkRed}{e-6}}}{3}%
            {e+7}{{\textcolor{dkRed}{e-7}}}{3}%
            {e+8}{{\textcolor{dkRed}{e-8}}}{3}%
            {e+9}{{\textcolor{dkRed}{e-9}}}{3}%
            {e-0}{{\textcolor{dkRed}{e-0}}}{3}%
            {e-1}{{\textcolor{dkRed}{e-1}}}{3}%
            {e-2}{{\textcolor{dkRed}{e-2}}}{3}%
            {e-3}{{\textcolor{dkRed}{e-3}}}{3}%
            {e-4}{{\textcolor{dkRed}{e-4}}}{3}%
            {e-5}{{\textcolor{dkRed}{e-5}}}{3}%
            {e-6}{{\textcolor{dkRed}{e-6}}}{3}%
            {e-7}{{\textcolor{dkRed}{e-7}}}{3}%
            {e-8}{{\textcolor{dkRed}{e-8}}}{3}%
            {e-9}{{\textcolor{dkRed}{e-9}}}{3}%
            {->}{{\textcolor{dkYellow}{->}}}{2}%
}

\lstdefinestyle{smedl}{
  style=smedlbase,
  float
}

\lstdefinestyle{smedlfixed}{
  style=smedlbase,
  style=fixed
}

\lstdefinelanguage{a4smedl}{
  morekeywords={system,import,monitor,as,event,syncset},
  morekeywords=[2]{int,float,double,char,string,pointer,thread,opaque},
  morekeywords=[3]{Id.,Param.},
  alsodigit={.},
  sensitive=true,
  morecomment=[l]{//},
  morecomment=[s]{/*}{*/},
  morestring=[b]"
}

\lstdefinestyle{a4smedlbase}{%
  language=a4smedl,
  commentstyle=\color{slate}\itshape,%
  keywordstyle=\color{dkBlue}\bfseries,%
  keywordstyle=[2]\color{dkGreen}\bfseries,%
  keywordstyle=[3]\color{dkRed},%
  stringstyle=\color{dkRed},%
  literate=*{0}{{\textcolor{dkRed}{0}}}{1}%
            {1}{{\textcolor{dkRed}{1}}}{1}%
            {2}{{\textcolor{dkRed}{2}}}{1}%
            {3}{{\textcolor{dkRed}{3}}}{1}%
            {4}{{\textcolor{dkRed}{4}}}{1}%
            {5}{{\textcolor{dkRed}{5}}}{1}%
            {6}{{\textcolor{dkRed}{6}}}{1}%
            {7}{{\textcolor{dkRed}{7}}}{1}%
            {8}{{\textcolor{dkRed}{8}}}{1}%
            {9}{{\textcolor{dkRed}{9}}}{1}%
            {$0}{{\textcolor{dkRed}{\$0}}}{2}%
            {$1}{{\textcolor{dkRed}{\$1}}}{2}%
            {$2}{{\textcolor{dkRed}{\$2}}}{2}%
            {$3}{{\textcolor{dkRed}{\$3}}}{2}%
            {$4}{{\textcolor{dkRed}{\$4}}}{2}%
            {$5}{{\textcolor{dkRed}{\$5}}}{2}%
            {$6}{{\textcolor{dkRed}{\$6}}}{2}%
            {$7}{{\textcolor{dkRed}{\$7}}}{2}%
            {$8}{{\textcolor{dkRed}{\$8}}}{2}%
            {$9}{{\textcolor{dkRed}{\$9}}}{2}%
            {\#0}{{\textcolor{dkRed}{\#0}}}{2}%
            {\#1}{{\textcolor{dkRed}{\#1}}}{2}%
            {\#2}{{\textcolor{dkRed}{\#2}}}{2}%
            {\#3}{{\textcolor{dkRed}{\#3}}}{2}%
            {\#4}{{\textcolor{dkRed}{\#4}}}{2}%
            {\#5}{{\textcolor{dkRed}{\#5}}}{2}%
            {\#6}{{\textcolor{dkRed}{\#6}}}{2}%
            {\#7}{{\textcolor{dkRed}{\#7}}}{2}%
            {\#8}{{\textcolor{dkRed}{\#8}}}{2}%
            {\#9}{{\textcolor{dkRed}{\#9}}}{2}%
            {=>}{{\textcolor{dkYellow}{=>}}}{2}%
}

\lstdefinestyle{a4smedl}{
  style=a4smedlbase,
  float
}

\lstdefinestyle{a4smedlfixed}{
  style=a4smedlbase,
  style=fixed
}

\lstdefinestyle{Cbase}{%
  language=C,
  commentstyle=\color{slate}\itshape,%
  keywordstyle=\color{dkBlue}\bfseries,%
  keywordstyle=[2]\color{dkGreen}\bfseries,%
  keywordstyle=[3]\color{dkRed},%
  stringstyle=\color{dkRed},%
  literate=*{0}{{\textcolor{dkRed}{0}}}{1}%
            {1}{{\textcolor{dkRed}{1}}}{1}%
            {2}{{\textcolor{dkRed}{2}}}{1}%
            {3}{{\textcolor{dkRed}{3}}}{1}%
            {4}{{\textcolor{dkRed}{4}}}{1}%
            {5}{{\textcolor{dkRed}{5}}}{1}%
            {6}{{\textcolor{dkRed}{6}}}{1}%
            {7}{{\textcolor{dkRed}{7}}}{1}%
            {8}{{\textcolor{dkRed}{8}}}{1}%
            {9}{{\textcolor{dkRed}{9}}}{1}%
            {$0}{{\textcolor{dkRed}{\$0}}}{2}%
            {$1}{{\textcolor{dkRed}{\$1}}}{2}%
            {$2}{{\textcolor{dkRed}{\$2}}}{2}%
            {$3}{{\textcolor{dkRed}{\$3}}}{2}%
            {$4}{{\textcolor{dkRed}{\$4}}}{2}%
            {$5}{{\textcolor{dkRed}{\$5}}}{2}%
            {$6}{{\textcolor{dkRed}{\$6}}}{2}%
            {$7}{{\textcolor{dkRed}{\$7}}}{2}%
            {$8}{{\textcolor{dkRed}{\$8}}}{2}%
            {$9}{{\textcolor{dkRed}{\$9}}}{2}%
            {\#0}{{\textcolor{dkRed}{\#0}}}{2}%
            {\#1}{{\textcolor{dkRed}{\#1}}}{2}%
            {\#2}{{\textcolor{dkRed}{\#2}}}{2}%
            {\#3}{{\textcolor{dkRed}{\#3}}}{2}%
            {\#4}{{\textcolor{dkRed}{\#4}}}{2}%
            {\#5}{{\textcolor{dkRed}{\#5}}}{2}%
            {\#6}{{\textcolor{dkRed}{\#6}}}{2}%
            {\#7}{{\textcolor{dkRed}{\#7}}}{2}%
            {\#8}{{\textcolor{dkRed}{\#8}}}{2}%
            {\#9}{{\textcolor{dkRed}{\#9}}}{2}%
            {=>}{{\textcolor{dkYellow}{=>}}}{2}%
}

\lstdefinestyle{C}{
  style=Cbase,
  float
}

\lstdefinestyle{Cfixed}{
  style=Cbase,
  style=fixed
}



\title{SMEDL Manual}
\author{University of Pennsylvania}

\begin{document}

\maketitle

%TODO Basic introduction to SMEDL

\autoref{part:lang} of this manual describes how to write a monitor. It discusses the features and syntax of the SMEDL specification language. \autoref{part:api} is about integrating monitors into your existing code. It describes the API for the generated code and the various transport mechanisms available for asynchronous connections.

\clearpage
\part{The SMEDL and A4SMEDL Monitoring Language}
\label{part:lang}

A SMEDL monitoring system is designed at two levels: the monitor level and
the architecture level. Monitor specifications describe a single monitor, that
is, a list of state variables, a list of events that the monitor consumes and
emits (known as ``imported'' and ``exported'' events), and a set of states,
transitions, and actions describing the state machine. Architecture
specifications describe how multiple monitors come together to form a
monitoring system, including which monitor specifications are involved, how
instances of the monitors are parameterized, and how imported and exported
events are directed.

Monitor specifications are written in \verb|.smedl| files, and there may be
multiple in a monitoring system. Architecture specifications are written in
\verb|.a4smedl| files and there is exactly one per monitoring system.

\section{SMEDL Specifications}
\label{sec:smedl}

A SMEDL specification describes a single monitor. This is where the actual
monitoring logic is defined. At its core, a SMEDL monitor is a set of state
machines. Transitions are triggered by events, with optional conditions and
associated actions. Monitors may also contain state variables, which the
conditions and actions can make use of.

\subsection{A Basic Monitor}
\label{subsec:smedl-basic}

\begin{lstlisting}[
  caption={A monitor that keeps a running total},
  label=lst:basic-mon,
  style=smedl
]
object Adder;

state:
  float accumulator = 0;

events:
  imported measurement(float);
  exported sum(float);

scenarios:
  main:
    idle
      -> measurement(val) {
        accumulator = accumulator + val;
        raise sum(accumulator);
      }
      -> idle;
\end{lstlisting}

\autoref{lst:basic-mon} gives an example of a simple monitor. This guide
will break it down piece-by-piece to familiarize you with the basic syntax of
SMEDL.

\begin{lstlisting}[firstnumber=1,style=smedlfixed]
object Adder;
\end{lstlisting}

This line declares the monitor specification name. It must be present at the
top of every monitor specification. The name can be any valid SMEDL identifier.
To be specific, an upper or lowercase letter, followed by zero or more
additional letters, numbers, or underscores.

This is the only line in the file that stands alone. The rest of the specification is split into three main sections, the start of each marked by a heading
keyword.

\begin{lstlisting}[firstnumber=3,style=smedlfixed]
state:
  float accumulator = 0;
\end{lstlisting}

This is the state variables section. State variables can be thought of as
global variables for the monitor. They will be useful for conditions and
actions (each of which are discussed later). There are seven types (and one
alias) available, listed in \autoref{tab:types1} along with the most similar C
type for each. This is the only optional section: If there is no need for state
variables, it may be omitted entirely.

\ctable[
  caption={The types available in SMEDL},
  label=tab:types1,
  mincapwidth=3in,
]{ll}{
  \tnote[a]{\texttt{double} is an alias of \texttt{float} for convenience, not a distinct type.}
  \tnote[b]{The \texttt{SMEDLOpaque} type is dicussed in %(TODO section from language reference).
  }
}{
  \toprule
  SMEDL & C \\
  \midrule
  \texttt{int} & \texttt{int} \\
  \texttt{float} & \texttt{double} \\
  \texttt{double}\tmark[a] & \texttt{double} \\
  \texttt{char} & \texttt{char} \\
  \texttt{string} & \texttt{char *} \emph{or} \texttt{char[]} \\
  \texttt{pointer} & \texttt{void *} \\
  \texttt{thread} & \texttt{pthread\_t} \\
  \texttt{opaque} & \texttt{SMEDLOpaque}\tmark[b] \\
  \bottomrule
}

\begin{lstlisting}[firstnumber=6,style=smedlfixed]
events:
  imported measurement(float);
  exported sum(float);
\end{lstlisting}

This is the event declarations section. Events are the input and output of the
monitor. Imported events come from outside the monitor and may trigger
transitions and actions. Exported events are generated within the monitor and
sent out. (There are also internal events, introduced in \autoref{subsec:smedl-advanced}.)

Events can carry parameters with them. These will be useful for conditions or
actions, discussed later. The parameters have predefined types, declared with
the event.

\begin{lstlisting}[firstnumber=10,style=smedlfixed]
scenarios:
  main:
    idle
      -> measurement(val) {
        accumulator = accumulator + val;
        raise sum(accumulator);
      }
      -> idle;
\end{lstlisting}

Finally, the scenarios section. This contains the actual monitor logic in the
form of state machines. There can be multiple scenarios, each representing one
state machine; however, this monitor only has one, named \texttt{main}.

Each scenario is defined as a list of transitions. Here, we only have one. From
the \texttt{idle} state, if there is a \texttt{measurement} event, the actions
in curly braces are performed, then it proceeds to the \texttt{idle} state
again.

The starting state for the first transition becomes the starting state for the
entire scenario.

Actions have a C-like syntax. There are five types of actions:
\begin{description}
  \item[\texttt{<state\_var> = <expr>}] assigns a value to a state
    variable.
  \item[\texttt{<state\_var>++}] increments a state variable.
  \item[\texttt{<state\_var>-{}-}] increments a state variable.
  \item[\texttt{raise <event>(<expr>[, <expr> \ldots])}] raises an exported orr
    internal event.
  \item[\texttt{<function>(<expr>[, <expr> \ldots])}] calls a helper function
    (see \autoref{TODO in language reference}).
\end{description}

Expressions are C-like as well and may use literals, event parameters, and
state variables.

\subsection{A More Advanced Monitor}
\label{subsec:smedl-advanced}

\autoref{lst:adv-mon} shows a monitor that ensures a light does not illuminate
until a button is pressed. It demonstrates some features of SMEDL that were not
present in the simple monitor above.

\begin{lstlisting}[
  caption={A monitor checking if a light stays off until a button press},
  label=lst:adv-mon,
  style=smedl
]
object WeakUntil;

state:
  int light = 0;
  int button = 0;

events:
  imported light_is(int);
  imported button_is(int);
  internal check();
  exported satisfaction();
  exported violation();

scenarios:
  input:
    idle -> light_is(status) {
        light = status;
        raise check();
      } -> idle;
    idle -> button_is(status) {
        button = status;
        raise check();
      } -> idle;

  verify:
    waiting
      -> check() when (!light && !button)
      -> waiting;
    waiting
      -> check() when (button) {raise satisfaction();}
      -> satisfied;
      else {raise violation();}
      -> violated;
    satisfied -> check() -> satisfied;
    violated -> check() -> violated;
\end{lstlisting}

This monitor has two scenarios, \texttt{input} and \texttt{verify}. These act
as separate state machines operating in parallel. Scenarios can share data and
interact with each other through state variables and internal events.

\begin{lstlisting}[firstnumber=10,style=smedlfixed]
  internal check();
\end{lstlisting}

This is an internal event declaration. Internal events can be raised like
exported events and trigger transitions like imported events, but do not leave
the monitor. They provide a way for actions in one scenario to send data to
or trigger a transition in another scenario.

As a result, a single imported event can cause a chain of internal and exported
events. This is known as a \emph{macro step}. There can only be one transition
per scenario per macro step. This is a guard against infinite loops.

Note that if a single event triggers a simultaneous transition in multiple
scenarios, either scenario is permitted to run first. However, any events
raised during a macro step are queued until all events raised before it are
finished being handled.

For example, given the scenarios in \autoref{lst:macro-steps}, if
\texttt{event\_a} were raised, the macro step would look like this:
\begin{enumerate}
  \item The transition in \texttt{scn\_a} would be triggered.
    \begin{enumerate}
      \item \texttt{event\_b} is raised. It is queued for later, since
        \texttt{event\_a} is still being handled.
      \item \texttt{my\_var} is set to \texttt{100}.
    \end{enumerate}
  \item \texttt{event\_b} triggers transitions in \texttt{scn\_b} and
    \texttt{scn\_bb}. These transitions could happen in either order, but we
    will do \texttt{scn\_b} first.
    \begin{enumerate}
      \item In \texttt{scn\_b}, \texttt{event\_d(100)} is raised. It is queued
        for later.
      \item In \texttt{scn\_bb}, \texttt{event\_c()} is raised. It is also
        queued.
    \end{enumerate}
  \item \texttt{event\_d} is popped back off the event queue. We will say this
    was an exported event, so this is the point where it would be emitted from
    the monitor.
  \item Next is \texttt{event\_c}, triggering a transition in \texttt{scn\_c}.
    \texttt{event\_e} is raised and queued.
  \item Finally, \texttt{event\_e} is dequeued. \texttt{scn\_a} also has a
    transition for this event, but since the macro step is not over, that
    transition is not taken.
\end{enumerate}

\begin{lstlisting}[
  caption={Example scenarios to demonstrate ordering within a macro step},
  label=lst:macro-steps,
  style=smedl
]
scenarios:
  scn_a:
    start -> event_a() {
      raise event_b();
      my_var = 100;
    } -> state_a;
    state_a -> event_e() -> start;

  scn_b:
    start -> event_b() {
      raise event_d(my_var);
    } -> start;

  scn_bb:
    start -> event_b() {
      raise event_c();
    } -> start;

  scn_c:
    start -> event_c() {
      raise event_e();
    } -> start;
\end{lstlisting}


%TODO Use the frontend monitor from multi_moving_average, or maybe a monitor
% for LTL. Actually, the latter is a good idea because it still makes sense
% as standalone. Go over more advanced features, like else, multiple scenarios,
% and internal events. Mention details like scenarios can only execute once per
% macro step and exported events can also act as internal events.

\subsection{SMEDL Language Reference}
\label{subsec:smedl-ref}

%TODO Start by describing the identifier format, then go into types, then
% go through sections one by one


%NOTE In an expression, == on strings normally compares the contents of the
% strings, but due to limitations in type verification, this does not happen
% when both sides of the == or != are helper functions that return strings.
% Similar for opaques.

%NOTE Opaque type does not allow any operations except == and !=. These
% operations do a byte-by-byte comparison with memcmp. If that is not
% acceptable, an alternative is to use a helper function that accepts two
% SMEDLOpaque and returns zero or nonzero.

\section{A4SMEDL Specifications}
\label{sec:a4smedl}

\subsection{A Basic Architecture}
\label{subsec:a4smedl-basic}

\subsection{A More Advanced Architecture}
\label{subsec:a4smedl-advanced}

\subsection{A4SMEDL Language Reference}
\label{subsec:a4smedl-ref}

%TODO Show a basic example going over the most common features. Then have sections on other features (like dynamic instantiation). Finish with a full reference (maybe in a separate chapter). Actually probably do this in sections.

%NOTE Using floats as identity parameters is experimental at best and should
% very likely be avoided. Hashing may not satisfy the invariant that if a == b,
% hash(a) == hash(b). And floats must compare exactly equal.
%NOTE Actually, I think the hashing is no longer a problem

%NOTE Using threads as identity parameters may result in undefined behavior.
% There is no portable way to compare `pthread_t`.

%NOTE See note in prev section on opaques. If they cannot be tested for equality
% with memcmp, then it is not safe to use them as monitor identities.

\clearpage
\part{The SMEDL Programming Interface}
\label{part:api}

The generated monitoring code is separated into multiple layers, ordered here
from lowest to highest:
\begin{description}
  \item[Monitor] contains the actual monitor state machines.
  \item[Local wrapper] manages instances of a single monitor specification and
    dynamic instantiation.
  \item[Global wrapper] manages all local wrappers in a single synchronous set
    and routes events between them.
  \item[Transport] handles asynchronous communication between global wrappers
    and the target system. This layer is only generated when the \texttt{-t}
    option is used with \texttt{mgen}.
\end{description}

The monitor (SMEDL) specification corresponds with the monitor level and the
architecture (A4SMEDL) specification corresponds with the three layers above
it.

\begin{shaded}
  Most programs will not need to interact with any layer except the transport
  layer directly. If not interfacing with SMEDL on a low level, it is safe to
  skip the sections on the monitor, local wrapper, and global wrapper layers.
\end{shaded}

\section{Common API Elements}
\label{sec:common}

\textbf{Header: \texttt{smedl\_types.h}}

\subsection{The \texttt{SMEDLValue} Type}
\label{subsec:smedlvalue}

The \texttt{SMEDLValue} type is used throughout the API for event parameters,
monitor identities, etc. It is defined in \autoref{lst:smedlvalue}. Each 
\texttt{SMEDLValue} contains a type \texttt{t} and a value \texttt{v}. The type
is a member of the \texttt{SMEDLType} enum and the value is a union. See
\autoref{tab:smedlvalue-union} for the correspondence between the SMEDL types
and union members.

\begin{lstlisting}[
  caption={The \texttt{SMEDLValue} and \texttt{SMEDLType} definitons},
  label=lst:smedlvalue,
  style=C
]
typedef enum {SMEDL_INT, SMEDL_FLOAT, SMEDL_CHAR,
    SMEDL_STRING, SMEDL_POINTER, SMEDL_THREAD,
    SMEDL_OPAQUE, SMEDL_NULL} SMEDLType;

typedef struct {
    SMEDLType t;
    union {
        int i;
        double d;
        char c;
        char *s;
        void *p;
        pthread_t th;
        SMEDLOpaque o;
    } v;
} SMEDLValue;
\end{lstlisting}

\ctable[
  caption={Correspondence between \texttt{SMEDLType} and union in
  \texttt{SMEDLValue}},
  label=tab:smedlvalue-union,
  mincapwidth=3in,
]{ll}{
  \tnote[a]{Must not be \texttt{NULL} and must be null-terminated.}
}{
  \toprule
  \texttt{SMEDLType} & Union member \\
  \midrule
  \texttt{SMEDL\_INT} & \texttt{int i} \\
  \texttt{SMEDL\_FLOAT} & \texttt{double d} \\
  \texttt{SMEDL\_CHAR} & \texttt{char c} \\
  \texttt{SMEDL\_STRING} & \texttt{char *s}\tmark[a] \\
  \texttt{SMEDL\_POINTER} & \texttt{void *p} \\
  \texttt{SMEDL\_THREAD} & \texttt{pthread\_t th} \\
  \texttt{SMEDL\_OPAQUE} &
    \hyperref[subsec:smedlopaque]{\texttt{SMEDLOpaque o}} \\
  \texttt{SMEDL\_NULL} & -- \\
  \bottomrule
}

The \texttt{SMEDL\_NULL} type is special---it does not correspond with an
actual SMEDL type. It is used internally in a couple places, but as far as the
API is concerned, there is only one significant use: wildcard parameters. In a
list of monitor identities, a \texttt{SMEDLValue} with \texttt{t} set to
\texttt{SMEDL\_NULL} is understood to be a wildcard. The \texttt{v} member is
ignored in such cases.

\subsection{The \texttt{SMEDLOpaque} Type}
\label{subsec:smedlopaque}

SMEDL's \texttt{opaque} type is meant to represent arbitrary-length chunks of
binary data. In that sense, it is somewhat similar to the \texttt{string} type.
The difference is that \texttt{string}s cannot contain null characters, since
they are null-terminated. The \texttt{opaque} type can, but that means its
size must be stored alongside it explicitly. Thus, we have the
\texttt{SMEDLOpaque} struct, defined in \autoref{lst:smedlopaque}.

\begin{lstlisting}[
  caption={The \texttt{SMEDLOpaque} struct},
  label=lst:smedlopaque,
  style=C
]
typedef struct {
    void *data;
    size_t size;
} SMEDLOpaque;
\end{lstlisting}

Every \texttt{SMEDLOpaque} struct must point to valid data; the \texttt{data}
pointer cannot be \texttt{NULL}\@. There is one exception to this: if the
\texttt{size} is zero, \texttt{data} may be NULL (or any other invalid
pointer). This goes for \texttt{SMEDLOpaque}s passed to callbacks as well.

Note that copies of opaque data are made during internal processing. This could
cause issues if there are any self-referential pointers within.

%NOTE Strings and opaques cannot point to NULL. Strings must always be
% null-terminated.

\subsection{The \texttt{SMEDLCallback} Type and Event Data}
\label{subsec:smedlcallback}

The SMEDL API naturally has several functions that pass event data from one
place to another. These functions all accept the same three parameters and do
not return anything, so to avoid repetition, these parameters are described
here.

\begin{description}
  \item[\texttt{SMEDLValue *identities}] is an array of the monitor identities.
    If there is no monitor (i.e. events from the target system) or the monitor
    has no identities, this can safely be \texttt{NULL}\@. Additionally, where
    wildcards are allowed, a \texttt{SMEDL\_NULL} value will represent one.
  \item[\texttt{SMEDLValue *params}] is an array of the event parameters. If
    the event has no parameters, this can safely be \texttt{NULL}\@.
  \item[\texttt{void *aux}] is a pointer that is passed through from imported
    events to exported events untouched. It can be used for provenance data.
\end{description}

The callbacks that SMEDL uses for exported events need to have the same
signature. The \texttt{SMEDLCallback} type (\autoref{lst:smedlcallback}) is a
function pointer for exactly that.

\begin{lstlisting}[
  caption={The \texttt{SMEDLCallback} type},
  label=lst:smedlcallback,
  style=C
]
typedef void (*SMEDLCallback)(SMEDLValue *identities,
    SMEDLValue *params, void *aux);
\end{lstlisting}

\subsection{Memory Ownership Conventions}
\label{subsec:mem-ownership}

The SMEDL API makes extensive use of arrays and pointers. Generally speaking,
the following principle holds:

%TODO This should probably be styled differently. One-element lists aren't
% really proper.
\begin{itemize}
  \item The caller is responsible for freeing its own memory. If a heap pointer
is passed as a parameter, the callee does not take ownership of it.
\end{itemize}

This applies to callbacks as well. Callbacks cannot assume pointers will remain
valid after they return, as the caller may (and often will) immediately free
them.

Exceptions to these rules (if any) will be noted with the particular function.

\section{Monitor API}
\label{sec:mon}

%TODO

\section{Local Wrapper API}
\label{sec:local}

%TODO

\section{Global Wrapper API}
\label{sec:global}

%TODO

\section{Transport Adapters}
\label{sec:transport}

%TODO

\subsection{RabbitMQ}
\label{subsec:rabbitmq}

%TODO

%\subsection{ROS}
%\label{subsec:ros}
%TODO

\subsection{File}
\label{subsec:file}

%TODO

%\section{PEDL API}
%\label{sec:pedl}
% At some point we may have an additional interface specifically designed for
% the target system to interface with. It would have functions named after the
% imported events (instead of channel names) and these functions would either
% call the global wrapper import API directly or generate RabbitMQ/ROS messages
% depending on whether synchronous or asynchronous transport is necessary. Such
% an interface would be documented in a section here.
\end{document}
